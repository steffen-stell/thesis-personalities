%Definition der wichtigsten Grundlagen (Schriftgröße, ein- oder zweiseitig, Papiergröße), kann man im Normalfall so lassen
\documentclass[12pt,oneside,a4paper]{article}

%Laden von "Paketen", wichtig für deutsche Sprache (Umlaute, Trennung usw) und das Einbinden von Grafiken
\usepackage[T1]{fontenc}
\usepackage{lmodern}
\usepackage{amsmath}
\usepackage{hyperref}
\usepackage[dvips]{graphicx}
\usepackage{array}
\usepackage{ntheorem}
\usepackage{tikz}
\theoremseparator{:}
\newtheorem{hyp}{Hypothesis}
\usepackage{pdfpages}
\usepackage[toc, page]{appendix}
\usepackage{dcolumn}
\usepackage{changepage}
\usepackage{url}
\usepackage[labelfont={bf,sf},font={small},%
  labelsep=period]{caption}
\usepackage{setspace}
\usepackage[style=apa, backend=biber]{biblatex}
\addbibresource{personalities.bib} %Import the bibliography file

\onehalfspacing
%Definition der Seite (schon etwas angepasst an die Anforderungen einer Uni-Arbeit)
%Ränder
\setlength {\oddsidemargin}{1,5 cm}
\setlength {\topmargin}{0 cm}
\setlength {\headsep}{0 cm}
\setlength {\headheight}{0 cm}
%Breite (nach rechts) und Länge (nach unten) des Texts auf der Seite
\setlength {\textwidth}{13 cm}
\setlength {\textheight}{25,0 cm}
\usepackage[american]{babel}
\usepackage{csquotes}
\usepackage{adjustbox}
\widowpenalty10000
\clubpenalty10000

\begin{document}

\title{\textbf{Master Thesis \\
\bigskip
\bigskip
Title of Thesis}}
\author{\\
		%\textbf{Steffen Stell} \\
		%01/954909 \\
        \includegraphics[scale=0.3]{fig/00_logo_kn.pdf}
		\\
    	Faculty of Politics, Law and Economics \\
		Department of Politics and Public Administration \\
		\\
		%1$^{st}$ Supervisor: Prof. Dr. Susumu Shikano \\
		%2$^{nd}$ Supervisor: PD Dr. Michael Herrmann
		}
\date{Konstanz, \the\year}

\maketitle
\thispagestyle{empty}

\newpage

% Bei doppelseitigem Druck leere Seiten einfügen
%
%\thispagestyle{empty}
%\mbox{}
%\newpage

\thispagestyle{empty}
\pagenumbering{roman}
\tableofcontents
\newpage

%\thispagestyle{empty}
%\mbox{}
%\newpage

\setcounter{page}{1}
%\input{001_abstract.tex}
\addcontentsline{toc}{section}{Abstract}
\newpage

\listoffigures
\newpage

\listoftables
\newpage

%\thispagestyle{empty}
%\mbox{}
%\newpage

\pagenumbering{arabic}
\section{Introduction}
\setcounter{page}{1}
%\input{01_introduction.tex}

\section{Literature}
%\input{02_literature.tex}

\section{Theory}
%\input{03_theory.tex}

\section{Data}
%\input{04_research_design.tex}
\subsection{Newspapers}
\subsubsection{Selection}
"Although local newspapers play an im-
portant role in the German media landscape, they often discuss the content that is
relevant to a locally restricted group of people only. Moreover, local newspapers tend to duplicate content that is relevant to a wider audience from a sister news-
paper within the same publisher, such as on topics regarding foreign affairs or
economic issues (Butenschön et al., 2017; Maute, 2011). " \parencite{Falck2019}

Text data collected from Lexis Nexis Uni. Selection of publication limited by licensed papers; FAZ, SZ and Handelsblatt cannot be accessed via Lexis Nexis Uni \parencite{LexisNexis2023}.

Ideologische Einordnung überregionaler Medien: Welt konservativ, FAZ mitte-rechts, SZ mitte-links, FR links, taz links-außen. Redaktionelle Linien sind in der Regionalpresse tendenziell weniger ausgeprägt, dies ist allerdings von der Wettbewerbssituation vor Ort abhängig. Mehr Wettbewerb = stärkere ideologische Prägung \parencite[pp. 129-130]{Maurer2006}. Despite changes in the newspaper market, the this spectrum remains accepted in the literature today \parencite[pp. 59-60]{Merkle2019}.

\section{Method}
\cite{Widmann2022} compare 3 different approches in measuring emotional sentiment in German political text. Main finding is that transformer models substantially outperform alternative approaches.


\section{Results}
%\input{05_results.tex}

\section{Discussion}
%\input{06_discussion.tex}

\section{Conclusion}
%\input{07_conclusion.tex}

\newpage
\printbibliography

\newpage

\section*{Appendix}
\addcontentsline{toc}{section}{Appendix}
%\input{09_appendix.tex}

\end{document}
